\documentclass{article}
\usepackage[a4paper, total={6in, 9in}]{geometry}
\usepackage[doublespacing]{setspace}		 
\usepackage{graphicx}        
\usepackage{lscape}		 
\usepackage{natbib}	
\bibpunct{(}{)}{;}{a}{}{,}		 	

\begin{document}

\title{\begin{LARGE} Hegemon and the Onset and Management of Maritime Claims\end{LARGE}}
\author{Jonghwan Han}
\date{}
\maketitle
  
\section*{Introduction}
\hspace{0.5cm}Does naval hegemon have a pacifying effect on maritime claims? On the 28th of October 2015, the United States dispatched their naval warships within 12 nautical miles of Subi reef in the South China Sea where China has built up an artificial island and has strongly argued its sovereignty. In response to this, China sent one of its most capable naval warships, so military tensions between the United States and China increased at that time. Why did the United States deploy its combatant ships in the South China Sea? Why did the United States care about the maritime claims in the South China Sea? In order to find answers for these questions, I focus on the role of naval hegemon over maritime claims. Based on the empirical results, I argue that naval hegemon has an incentive to settle maritime claims in a peaceful way. Therefore, the behaviors of the United States Navy in the South China Sea could be understood as the role naval hegemon to maintain “good orders” in the sea. 

Maritime claims have occurred since the ancient times, but currently, increasing demands for natural resources (oil, natural gas, etc.) and fishing have gradually increased the possibility of maritime claims (Nyman 2013). When comparing to other types of territorial claims, maritime claims have inherent problems: uncertain borderlines. As Ostrom (1990) mentions, it is clear to divide the land into parcels and assign individual rights to hold, while establishing boundaries of nonstationary resources, especially water is difficult and unclear. After the adoption of United Nations Convention on the Law of the Sea (UNCLOS) in 1982 which proclaimed criteria to set up boundary lines of territorial waters and Exclusive Economic Zones (EEZs)\footnote{UNCLOS defines that territorial water is a belt of coastal waters extending at most 12 nautical miles and EEZ stretches from the baseline out to 200 nautical miles from its coast. Within the EEZ a state has special rights regarding the exploration and use of marine resources including energy production from water and wind.} , the discordance among states regarding maritime territories has been intensified. According to the Issue Correlates of War (ICOW) dataset, 59 of 143 maritime claims (41.2 percent) are ongoing, which is four times larger than land territorial claims in progress. Furthermore, the rate of reaching an agreement of maritime claims has been decreasing. Between 1970 and 1982 approximately eight maritime claim agreements were concluded each year. However, since 1982, the average number of agreements concluded annually has leveled at just over five(Donaldson and Williams 2005). 

From the perspective of economy and security issue, maritime areas are substantively significant for states. First, in terms of security, conflicts over territory are more likely to involve military forces and escalate to war than other issues (Diehl 1999). As a type of territorial claims, maritime claims also have an inherent characteristic of potentially powerful violence. Second, in terms of economic issue, not only natural resources such as oil/natural gas in the sea play an important role in determining states’ economic situation, but also transportations of commodities and resources via sea lanes. Therefore, good orders in maritime areas should be controlled and upheld by specific mechanisms. Naturally, it is reasonable to assume that frequent occurrences of maritime claims and difficulties in reaching an agreement over the claims result from the malfunctions of these mechanisms. In this paper, I mainly focus on one of the mechanisms, hegemonic power.  

From the perspective of economy and security issue, maritime areas are substantively significant for states. First, in terms of security, conflicts over territory are more likely to involve military forces and escalate to war than other issues (Diehl 1999). As a type of territorial claims, maritime claims also have an inherent characteristic of potentially powerful violence. Second, in terms of economic issue, not only natural resources such as oil/natural gas in the sea play an important role in determining states’ economic situation, but also transportations of commodities and resources via sea lanes. Therefore, good orders in maritime areas should be controlled and upheld by specific mechanisms. Naturally, it is reasonable to assume that frequent occurrences of maritime claims and difficulties in reaching an agreement over the claims result from the malfunctions of these mechanisms. In this paper, I mainly focus on one of the mechanisms, hegemonic power. 

Beside, for naval hegemon, absolute control of all maritime areas to uphold maritime status quo is impossible due to the vast area to cover. Neither Britain nor the United States which are regarded as naval hegemon have ever totally dominated the world so far (Keohane 1984). Therefore, the influence of naval hegemon to maintain orders in the sea varies based on some factors. I think that one of these factors is whether one or both claimants are allied with naval hegemon because of the entrapment problem. In order to avoid an unintended involvement into a conflict due to alliance (entrapment problem), naval hegemon is more likely to pay more attention to deter new maritime claims and settle maritime claims peacefully when its allies are related to maritime claims. 

In order to examine the role of naval hegemon on maritime claims, I organize this paper with four parts. First, I explain the hegemonic stability theory in order to provide a theoretical background of this research. In the second session, explanations regarding naval hegemon and maritime claims, especially, the importance of naval hegemon to explain maritime claims is mainly discussed. The next parts are research design and empirical results. I use the rare event logit model for maritime claim onset and the negative binomial regression for the peaceful settlement attempts over maritime claims. Lastly, I mention some limitations of this research and future research directions.

\section*{Hegemonic stability theory}
\hspace{0.5cm}As opposed to the balance of power theory which argues that the equality of power between states contributes to peace, advocates of the hegemonic stability theory insist that the international order shaped by a single dominant state is stable, and this situation occurs frequently. Two dominant theories, power transition and long cycle theory, provide plentiful explanations regarding the secured and stable order under hegemonic power. 

First, proponents of power transition theory argue that strong concentration of power in a single country is historically and empirically common and peaceful. Organski (1958) described that a powerful state frequently dominates international systems based on its capability to enhance the security of system through setting up political and economic structures and rules of behaviors. The dominant power makes and sustains the international order in order to further its national interests and spread specific kind of regimes and norms. For example, a democratic hegemon is more likely to create democratically operating system. Once the order is established, the dominant power preserves the order and maintain stability in the system, since it is conducive to its long-run benefits. It does not mean that all other states are satisfied with the existing order forced by the dominant state. Some states are disadvantaged by the status quo, and these dissatisfactions make them change the prevailing order. However, if the dominant power remains preponderant, challenging states have no incentive to counter the hegemon, and follow the existing rules (Kugler and Lemke 2000). In other words, under the strong hegemonic order, some states are satisfied with the status quo, of course, follow the existing order to get security and economic benefits, while other dissatisfied states are usually week to counter the dominant state (Organski and Kugler 1980). Therefore, war is least likely when the dominant state’s power exceeds that of others because the others do not possess enough power to initiate war due to low probability of success (Kadera 2001). However, due to different growth rates and technological innovations, the rise and fall of dominant power as well as emergence of strong challengers are inevitable. More specific explanation is discussed below.

Next, long cycle theory explains that the occurrence of global war is related to the rise and fall of global leadership. Rasler and Thompson (1994) suggest the repeated four stages to explain the relationship between the wax and wane of global leadership and the occurrence of global wars. Among the four phases, the execution phase begins when a single state emerges as a dominant power after a global war. In this stage, the dominant power sets the rules of behaviors and spread norms that reflect its national interests. In addition, in this phase, the distribution of power remains stable, and no states want to exit or counter the existing order in the execution phase. In common with the power transition theory, stable governance under the global leadership is not static. In order words, governance from global leadership hinges on the concentration of capability of global reach which is fluctuated by technological innovations. As the dominant power falls and other states rises, the questions about the existing order initiated by the incumbent leader arise (Colaresi 2001).

In sum, theories about hegemonic stability insist that presence of preponderant hegemonic power leads to stability in the system, since states which are advantaged by the existing order naturally follow the system to further their interests, and the states which are disadvantaged by the status quo have no choice but to subordinate the current rule due to huge power disparities. However, strength of the dominant power is dynamic, not static. Uneven patterns of growth due to industrialization and technological innovations cause not only the decline of dominant power in the international politics, but also subsequent challenges to the dominant state’s position by other states undergoing dramatic development. Therefore, the decline of dominant power leads to some leeway for dissatisfied states to further their national interests, and the wane of hegemon is associated with disorders in the system and undesirable consequences for other states (Sinidal 1985).  

There would be no exceptions on the issue of maritime areas. Based on the hegemonic stability theories, I would argue that preponderant naval hegemon makes and maintains orders in the sea in order to maximize its national interests, while other states are more likely to follow and subordinate the orders forced by naval hegemon due to huge power disparities. However, the decline of naval hegemon provides dissatisfied states with some opportunities to express their discontents, and leads to more frequent maritime disorders (maritime claims) in the sea. In order to evaluate these expectations, it is necessary to understand maritime claims and naval hegemon. 


\section*{The onset and management of maritime claims}
\hspace{0.5cm}“Maritime claims” is defined as an explicit contention between two or more states over the access or usage of maritime areas. To be specific, maritime claims occur when official representatives of two state governments contest the sovereignty or usage of maritime space. In this definition, the “official representatives” include a variety of entities such as foreign minister and representative individuals of a state, and other legitimate political or military officials on behalf of the state’ governments. In addition, maritime claims do not require any specific form of contention. In other words, deploying one or both sides militarized forces over the maritime claims is not required to define maritime claims (Hensel et al. 2008). Usually, claims about islands or continental shelves, access to fishing areas, demarcating sea borderlines such as territorial waters, EEZs are the common examples of maritime claims.

Due to economic and political importance of maritime areas, several research investigates maritime claims, especially they mainly focus on the onset and management of maritime claims based on international regimes (UNCLOS, EEZs) and its issue salience. First, in terms of maritime claim onset, Nemeth et al. (2014) argue that UNCLOS deters the initiation of new maritime claims, since it provides claimants with clear information regarding how to set up and manage maritime borders such as territorial waters, EEZs, and continental shelfs. Based on the empirical results, they find that both monadic and dyadic UNCLOS memberships reduce the chance of new maritime claims. Similar to this, Mitchell and Owsiak (forthcoming) examine the role of UNCLOS and maritime claim onset. They find that since UNCLOS is highly legalized and judicialized regime to settle maritime claims that influence state’s bargaining processes, UNCLOS members are less likely to challenge the status quo by initiating new maritime claims. 

Regarding the management of maritime claims, Nemeth et al. (2014) evaluate the influences of UNCLOS and EEZs on peaceful and militarized management of maritime claims by analyzing 3,231 dyad-year data. This research concludes that EEZs promote more frequent bilateral negotiations to produce agreements between disputing states. UNCLOS is also successful at bringing third-parties to the conflict management table, and conducive for preventing new disagreements over maritime areas. In term of pacifying effects of UNCLOS, Mitchell and Owsiak (forthcoming) also find that UNCLOS members are less likely to use military forces and more likely to attempt to peaceful settlement strategies over maritime issues.  
 
Issue salience is an another important part to explains maritime claims. Hensel (2001) and Hensel et al. (2008) analyze the relationship between salience of issues and state’s course of action. He reveals that state’ options between peaceful and militarized actions for managing territorial issues are systematically influenced by the salience of issue. To put it concretely, more salient territorial claims are more likely to result in bilateral negotiations or militarized actions (Hansel 2001). Furthermore, states are prone to use both militarized conflicts and peaceful methods when the dispute is more salient (Hensel et al. 2008). An interest thing to Hensel’s research (2001) is that democratic dyads have no significant impacts on the militarized conflicts in territorial disputes. In other words, democracy is not a panacea cure for the problem of territory. However, Mitchell (2002) has an opposite conclusion. She analyzes a variety of attempts of states to settle territorial claims peacefully by examining 114 dyadic territorial claims. She demonstrates that when the proportion of democracies increases in international system, the probability of third-party settlements increases. Especially, when compared to democratic dyads, the effects of democracy in nondemocratic dyads are more noticeable than those of democratic dyads. 

With previous literatures about maritime claims done, I could confirm that research about maritime claims needs be improved in two ways. First, it has to be supplemented from the perspective of realism. There are several studies which show the effects of UNCLOS or EEZs over maritime claims, while there are lack of research that analyzes maritime claims with traditional concept of power. It does not mean that these previous literatures ignore the concept of power. Even though they admit that relative power or capabilities between claimants have strong influences on maritime claims, they regard relative capabilities as a control variable. Second, there are not any research which analyzes maritime claims based on naval power. As Nyman (2013) argues, international maritime disputes are conducted by claimants’ official navy. In addition, although previous research includes state capabilities or military power to explain maritime claims, most of them use the Composite Index of National Capability (CINC) scores, not naval power itself. Therefore, it is necessary to investigate the role of naval power, especially naval hegemon, over maritime claims. 

Why is naval hegemon important? Is worth to research naval hegemon which only focuses on naval power to explain the important issue (maritime claims) in world politics? Before I answer these question, Modelski and Thompson (1988) provide a brief answer. They argue that there can be no global systems without global reach based on powerful navy, and naval power is an essential component of world order because of what navy, and navies alone, can do in the sea.

Prior to explain naval hegemon, it is necessary to examine the definition of “naval power” and “sea power”. Mahan coined this terminology; sea power, but he did not exactly define it. In his book, he identified six critical elements  of sea power and concluded that being a great power meant being a sea power (Gooch 1989). In a broad definition, sea power encompasses military as well as private fields which are related to the sea. Like its constituent “power”, it has the meaning of both an input and an output. Sea power as an input means navies, coastguards, and civilian maritime industries, while sea power as an output means capacity to influence behaviors of other states (Till 2013). Therefore, naval power means military capacity of sea power. For the purpose of this article, I will define naval power as a major naval strength that is forces capable of using and exercising control over the sea. Next, it is important to determine how to define naval hegemon. In Modelski and Thompson’s research (1988), they define naval hegemon as a state which has over 50 percent of the total warships in the world in order to project its power over the sea. However, the main point of Modelski and Thompson’s definition of naval hegemon is not the specific criteria (50 percent of total warships), but the characteristic of ocean-going power projection capabilities. Mahan (1987) argues that world powers, Portugal in the 15C, Netherlands in the 16C, Britain in the 17 ~ 19C, and the United States after 20C, have been great naval powers due to their long-range power projection capabilities. Therefore, based on the time period of this research (1900 ~ 2001), the Royal Navy (Britain) from 1900 to the World War II and the United States Navy from the World War II to 2001 are designated as naval hegemon. 

As Modelski and Thompson (1998) and Mahan (1987) argue, world leadership requires capacity for global reach, and naval power is the optimal mean for that purpose. Mearsheimer (2014) argues that there has been no global hegemon in the history, since sea blocks the power projection. Conversely speaking, if a state has mighty naval power to overcome long-distance power projection problem, they could be a global hegemon. Therefore, naval hegemon based on mighty naval power has played an important role in establishing the rules and upholding the status quo structures governing relations among states in the sea. 

Over the past 200 years, Great Britain and later the United States have been the naval hegemon through their dominance of the seas, and they have been able to shape the political and economic order of the world (Murphy and Yoshihara 2015). They have tried to maintain maritime status quo by deterring emergence of new challengers in the sea. Beside, naval hegemons also have led and influenced other states behaviors regarding the issues in maritime areas. For example, Truman’s proclamation regarding extended territorial claims encouraged other states to do the same way (Mitchell and Owsiak forthcoming). In addition, even though the United States hasn’t ratified the UNCLOS, the United States has been a leading influence since its inception. After 9/11, the United States offered a number of binding security measures, and some of them were subsequently enacted the International Maritime Organization (IMO) which is the competent international organization to support UNCLOS, include 2005 counter-terrorism amendments to the 1988 Convention for the Suppression of Unlawful Acts against the Safety of Maritime Navigation (Pedrozo 2010). Therefore, the role of naval hegemon based on mighty naval power is an important factor to explain issues in the maritime areas as well as events in world politics.

\section*{Influence of Naval hegemon on maritime claims}
\hspace{0.5cm}When considering the geographic characteristics of maritime areas, naval power would be a core constitutive element in order to deal with issues in maritime areas. The Army cannot project its forces without naval platforms because the sea stops the Army’s power projection (Mearsheimer 2014). The Air Force also has some limitations in responding to issues in the sea due to its limited operation time and area. In addition, as Corbett (1911) mentions, there are inherently different conditions between war at land and at sea: primary objective, uncertainty of enemy’s possible movements, additional duty of sea battle to protect commerce, etc. 

The wax and wane of naval hegemon influences its enforceable power to maintain status quo in the maritime areas. Naturally, the powerful naval hegemon could provide enough naval forces to sustain the stable situations and minimize operational vacuums in the maritime areas. In that case, states which are involved in the maritime claim cannot help following the established norms and rules, so they are less likely to conduct dangerous military operations and more likely to attempt peaceful actions to settle the claims. On the contrary, the decline of naval hegemon leads to lack of actual naval forces in maintaining the status quo and enforcing the established rules and norms in the maritime areas. Consequently, under the influences of strong naval hegemon, states are less likely to challenge the maritime status quo and more likely to settle the issues peacefully. So, the following hypotheses can be made: 
\\

\emph{Hypothesis 1: As the capability of naval hegemon increases, states are less likely to initiate new maritime claims.}
\\

\emph{Hypothesis 2: As the capability of naval hegemon increases, states are more likely to attempt to peaceful settlement strategies over maritime claims.}
\\

However, maritime areas are very huge. It means that even though naval hegemon has powerful navy, it does not mean that naval hegemon’s naval power presents the same effects to all maritime areas. In other words, even though the naval hegemon maintains a mighty naval power which overwhelms other states’ naval power, it is nearly impossible to control all maritime areas at the same time. Naturally, the efforts of naval hegemon to maintain the maritime status quo vary based on some factors. Alliance would be an importance mechanism to explain variations of influence of naval hegemon. When state A is allied with state B, state A is faced with entrapment problem which means being dragged into a conflict for its ally. When state A heavily depends on a powerful ally, the likelihood of entrapment problem increases for the powerful state, since state A is more likely to act aggressively due to the expectation of supports from its strong ally (Snyder 1984). Similar to this, if one or both claimants are allied with naval hegemon, the likelihood of being dragged into a potential conflict for naval hegemon increases. Therefore, naval hegemon is prone to pay more attention to deter new maritime claims and to encourage claimants to settle the maritime claims peacefully when its allies are involved in the maritime claims. So, the following hypothesis can be made:
\\

\emph{Hypothesis 3: When one or both claimants over maritime claims are allied with naval hegemon, the likelihood of maritime claims onset decreases.}
\\

\emph{Hypothesis 4: When one or both claimants over maritime claims are allied with naval hegemon, the likelihood of peaceful settlement attempts increases.}
\\

\section*{Research Design}
\emph{Dependent variables}

Two dependent variables, maritime claim onset and peaceful settlement attempts are measured based on the ICOW claim dyad year dataset. This dataset is an observational data that includes maritime claims from 1900 to 2001 in the Western Hemisphere and Europe (the data regarding Asian and Africa currently underway). This dataset includes annual information on each dyadic claim such as claim salience, information on peaceful and militarized settlement attempts, and summaries of recent interactions over the claim, etc. Unit of analysis is dyad-years, because most maritime claims are based on the actions and reactions of two states. 

The first independent variable, maritime claim onset is a dichotomous variable which shows whether the maritime claim begins in a given dyad-year or not. “1” mean that the claim begins the given dyad year, “0” otherwise. Figure 1 shows the distribution of maritime claim onset. Only 5 percent of observations are coded 1. In this cause, using the logit model can underestimate the probability of rare events (King 2001). In order to compensate this problem, I use the rare event logit model for maritime claims onset.
\\

\begin{figure}[h]
\centering
\includegraphics[scale=0.115]{graph1}
\caption{The distribution of maritime claim onset}
\end{figure}

Another dependent variable, peaceful settlement attempts, is measured by the total number of peaceful (bilateral and third party) settlement attempts that begin in the given dyad year. Figure 2 shows the distribution of peaceful settlement attempts. 

\begin{figure}[h]
\centering
\includegraphics[scale=0.115]{graph2}
\caption{The distribution of peaceful settlement attempts}
\end{figure}

The distribution of peaceful settlement attempts presents that most (95 percent) of observations are coded 0 which means none occurrence of peaceful settlement attempts. Therefore, based on the distribution (over-dispersion) and characteristic of the dependent variable (count variable), I use the negative binomial regression for peaceful settlement attempts.
\\

\emph{Independent variables} \\
   First, in order to measure capability of naval hegemon, it is necessary to specify how to measure naval power. Traditionally, Modelski and Thompson’s method is widely used to measure naval power. They argue that even though many of conventional capability attributes (military expenditures, personnel, etc.) influence strength of naval power, they have inherent limitations to be measured. For example, there are lack of exact data regarding naval expenditures due to different criteria to ascertaining actual expenditures among countries. In terms of personnel, since a main mean of naval warfare is not personnel but warships, the number of personnel is less meaningful. Therefore, Modelski and Thompson (1988) mainly consider the purpose of navy (ocean-going power projection), and measure naval power with total number of warships. However, the important problem of this measuring method is that it does not reflect the quality of naval power and its relative preponderance against other states. In order to supplement the quality dimension of naval power and relative strength, I measure naval power based on the percent of total tonnage of warships in the world, because usually heavier warships are more likely to equip more and better weapon systems and have longer operation periods. Therefore, capability of naval hegemon is measured by the percent of total tonnage of warships in a given year. 
   
   As I discussed earlier, since the Royal Navy and the United States Navy are regarded as naval hegemon, capability of naval hegemon is calculated based on the percent of total tonnage of warships in Royal Navy from 1900 to 1943 and those of the United States Navy from 1944 to 2001. In order to measure the percent of total tonnage of warships, I use Crisher Brian and Mark Souva’s naval dataset (2014) that contains total number and tonnage of warships from 1865 ~ 2011. 

Another independent variable, alliance with naval hegemon is measured from the Alliance Treaty Obligations and Provisions (ATOP) datasets. “1” means that the one or both claimants have any type of alliance with the naval hegemon, “0” otherwise. 
\\

\emph{Control variables}

First, I make three kinds of dichotomous variables regarding the effects UNCLOS. In order to examine systemic effect of UNCLOS, I make a UNCLOS adoption which shows the absence and presence of UNCLOS. Since UNCLOS has been accepted by states after 1982 and forced to accept for states after 1994, years after 1982 are coded “1” which means the presence of UNCLOS, “0” otherwise. In addition, in order to identify the dyadic level effects of UNCLOS, I generate UNCLOS1 and UNCLOS2 which show whether one or both claimants adopted UNCOS or not. UNCLOS1 is coded “1” if only one of claimants agree to UNCLOS, “0” otherwise. UNCLOS2 is coded as “1” if both claimants agree to UNCLOS, “0” otherwise. 

Second, relative capabilities between states influence states decisions to challenge states quo or to use peaceful settlement strategies over maritime claims, since relative powers are directly related to states’ bargaining power. Therefore, if there are huge disparities of power between claimants, the stronger state is more likely to initiate a new marmite claim and less likely to use peaceful settlement strategies over maritime claims. I use the Composite Index of National Capabilities (CINC) on the Correlates of War datasets (Singer, Bremer, and Stuckey 1972) to measure relative capabilities. Relative capabilities are measured by dividing the stronger state’s CINC score by the sum of stronger and weaker state’ CINC score. It ranges from 0.5 to 1.0, and 1.0 means stronger state consists of all dyad’s capabilities (Nemeth et al. 2014).

Third, maritime issue salience would affect state’s courses of action over maritime issues. As Hensel et al. (2008) argues, states are prone to use both militarized conflicts and peaceful methods when the claim is more salient. Therefore, I anticipate that state is more likely to apply to peaceful settlement strategies and to initiate new maritime claims when maritime issues are more salient. The ICOW maritime data measures the salience of issue on the basis of 6 indicators; (1) whether the maritime zone is associated with claimants’ homeland or not, (2) the known/believed presence of natural resource, (3) a strategic location of the maritime claim, (4) the presence of fishing resources, (5) migratory fishing stocks across to the maritime zone, (6) the connection of maritime claims to territorial claims (ICOW Maritime Codebook). I generate two variables to show the salience of issue. First, on the basis of economic importance, I include a dichotomous variable which shows the known/believed presence of resource within the maritime claim zone. “1” means the presence of resources, “0” otherwise. Other indicators for issue salience are combined to generate other issue salience variable which range from 0 to 10.

Fourth, shared memberships in international institution is also included in the models. Shared memberships is measured through the Multilateral Treaties of Pacific Settlement (MTOP) dataset which is measured as the number of memberships of multilateral peaceful institution that both claimants signed and ratified (Hensel et al. 2008). I measure this variable by using the ICOW dyad year dataset. I expect that if claimants share more multilateral peaceful institution memberships, they are less likely to initiate new maritime claims and more likely to use peaceful settlement skills over maritime claims.

Fifth, I examine the impact of claimant’s democracy level by using joint democracy variable which shows whether claimants are both democracy or not. “1” means both claimants’ POLITY IV index of institutionalized democracy scores (Jaggers and Gurr 1995) are six or higher, “0” otherwise. If both claimants are established democracy, they are less likely to challenge the maritime status quo and more likely to conduct peaceful management skills. 

Lastly, I control for the influences of recent militarized disputes. As Diehl and Goertz (2000) argue, recent militarized disputes over territorial issue raise the probability of future usage of militarized forces over the same issues. I adopt Nemeth et al.’s (2014) way to calculate recent militarized disputes. Based on the total number of militarized disputes between claimants over the maritime claim in the 10 years prior to the current year, weighted to reflect declining effects over time. Militarized disputes in the year before the claim contribute a value of 1.0, and this weight declines by 10 percent each year.

\section*{Empirical results}

\hspace{0.5cm}For the first dependent variable, maritime claim onset, I apply to the rare event logit model. My hypotheses regarding maritime claim onset suggest that the likelihood of maritime claim onset decreases when the capability of naval hegemon increases and one or both claimants are allied with naval hegemon. However, I do not find any signficant evidence to support my hypothesis 1,\emph{as the capability of naval hegemon increases}, states are less likely to initiate new maritime claims, and hypothesis 3, \emph{when one or both claimants over maritime claims are allied with naval hegemon}, the likelihood of maritime claim onset increases.

Next, I examine how the capability of naval hegemon influences peaceful attempts to settle maritime claims by using the negative binomial regression model. In table 1, model 1 shows empirical results and figure 3 presents a graph for coefficients. Capability of naval hegemon has a significant positive effect on peaceful settlement attempts). It means that under mighty naval hegemon, states are more likely to settle the claims peacefully. This result supports my hypothesis 2, as the capability of naval hegemon increases, states are more likely to attempt to peaceful settlement strategies over maritime claims. 


\begin{table}[!htbp] \centering 
  \caption{The influence of naval hegemon on peaceful settlement attempts over the maritime claims} 
  \label{} 
\begin{tabular}{@{\extracolsep{5pt}}lD{.}{.}{-3} D{.}{.}{-3} } 
\\[-1.8ex]\hline 
\hline \\[-1.8ex] 
 & \multicolumn{2}{c}{\textit{Dependent variable:}} \\ 
\cline{2-3} 
\\[-1.8ex] & \multicolumn{2}{c}{peaceful settlement attempts} \\ 
\\[-1.8ex] & \multicolumn{1}{c}{(1)} & \multicolumn{1}{c}{(2)}\\ 
\hline \\[-1.8ex] 
 Capability of naval hegemon & 0.030^{***} & 0.008 \\ 
  & (0.011) & (0.015) \\ 
  Alliance with naval hegemon & 1.344^{***} & -0.716 \\ 
  & (0.344) & (1.150) \\ 
  UNCLOS & 0.031 & 0.351 \\ 
  & (0.346) & (0.392) \\ 
  UNCLOS1 & 0.204 & 0.150 \\ 
  & (0.426) & (0.429) \\ 
  UNCLOS2 & 1.014^{*} & 0.880 \\ 
  & (0.564) & (0.567) \\ 
  Relative capabilities & -1.250^{*} & -1.190 \\ 
  & (0.752) & (0.761) \\ 
  Resource & -0.686^{***} & -0.752^{***} \\ 
  & (0.237) & (0.239) \\ 
  Other salience & 0.102 & 0.122^{*} \\ 
  & (0.069) & (0.071) \\ 
  Shared memberships & -0.148^{***} & -0.161^{***} \\ 
  & (0.046) & (0.047) \\ 
  Joint democracy & 0.492^{*} & 0.490^{*} \\ 
  & (0.253) & (0.257) \\ 
  Recenty militarized disputes & 1.106^{***} & 1.211^{***} \\ 
  & (0.237) & (0.237) \\ 
  interaction &  & 0.043^{**} \\ 
  &  & (0.022) \\ 
  Constant & -4.080^{***} & -3.319^{***} \\ 
  & (0.933) & (0.973) \\ 
 \hline \\[-1.8ex] 
Observations & \multicolumn{1}{c}{3,084} & \multicolumn{1}{c}{3,084} \\ 
$\theta$ & \multicolumn{1}{c}{0.042^{***}  (0.006)} & \multicolumn{1}{c}{0.042^{***}  (0.006)} \\ 
Akaike Inf. Crit. & \multicolumn{1}{c}{1,518.938} & \multicolumn{1}{c}{1,517.152} \\ 
\hline 
\hline \\[-1.8ex] 
\textit{Note:}  & \multicolumn{2}{r}{$^{*}$p$<$0.1; $^{**}$p$<$0.05; $^{***}$p$<$0.01} \\ 
\end{tabular} 
\end{table}

\begin{figure}[h]
\centering
\includegraphics[scale=0.6]{R3}
\caption{The influence of naval hegemon on peaceful settlement attempts over the maritime claims}
\end{figure}

Figure 4 shows a substantive effect of naval hegemon on peaceful settlement attempts. Increasing capability of naval hegemon from one standard deviation below the mean to one standard deviation above the mean increases the predicted number of peaceful attempts by 225 percent.

\begin{figure}[h]
\centering
\includegraphics[scale=0.09]{graph4}
\caption{Predicted number of peaceful settlement attempts varying based on capability of naval hegemon}
\end{figure}

Alliance with naval hegemon is also positively related to peaceful settlement strategies and statistically significant which means that if one or both claimants are allied with naval hegemon, they are more likely to attempt to peaceful options over maritime claims. This result supports my hypothesis 4, when one or both claimants over maritime claims are allied with naval hegemon, the likelihood of peaceful settlement attempts increases. In other words, due to the entrapment problem, the United States and Britain (naval hegemon) pay more attention to settle the maritime claims peacefully where its allies are involved as in the case of the South China Sea. 

Figure 5 shows a substantive effect of alliance with naval hegemon on peaceful settlement attempts. When one or both claimants are allied with naval hegemon, the predicted number of peaceful settlement attempts increase by 300 percent when comparing to the claims where no claimants are allied with naval hegemon.

\begin{figure}[h]
\centering
\includegraphics[scale=0.1]{graph5}
\caption{Predicted number of peaceful settlement attempts varying based on alliance with naval hegemon}
\end{figure}

The effects UNCLOS are mixed. Once both states are the member of UNCLOS, the likelihood of attempting peaceful settlements increase, although there are insignificant effects for UNCLOS adoption and one-sided UNCLOS membership. This indicates that maritime claims are successfully managed when two claimants jointly agree to UNCLOS and view it as a proper mean to solve conflicts.

Relative capabilities are negatively related as expected. It indicates that huge disparities of capabilities between claimants deter the stronger state to apply peaceful settlement attempts. Joint democracy also has a significant effect which means that if both claimants are established democracies, they are more likely to settle the claims peacefully. These findings for relative capabilities, joint democracy are consistent with the another research (Mitchell and Owsiak forthcoming) regarding peaceful settlement attempts. However, recent militarized disputes have positive effects and statistically significant, and the presence of resources in the claimed area leads to less peaceful settlement attempts, while other issue salience is not significant. Shared memberships are negatively related to the peaceful attempts to settle the claims which is opposite with the expected direction. 

In sum, although capability of naval hegemon does not show statistically significant result on maritime claim onset, naval hegemon is effective to change state behaviors to settle maritime claims since empirical result shows that as capability of naval hegemon increases, claimants over maritime claims are more likely to attempt to peaceful settlement strategies. In addition, when allies of naval hegemon are involved in maritime claims, states are more likely to attempts peaceful settlement attempts.

In table 1, model 2 shows that the interaction term between capability of naval hegemon and alliance with naval hegemon has a positive and significant effect on peaceful settlement attempts. It means that the pacifying effect of naval hegemon increases when one or both claimants over maritime claims is allied with naval hegemon. Figure 6 presents a substantive effect. When one or both claimants are allied with naval hegemon, increasing capability of naval hegemon from one standard deviation below the mean to one standard deviation above the mean increases the predicted number of peaceful attempts by 650 percent.

\begin{figure}[h]
\centering
\includegraphics[scale=0.1]{graph6}
\caption{Predicted number of peaceful settlement attempts varying based on capability of naval hegemon and alliance with naval hegemon}
\end{figure}

\section*{Conclusion}
\hspace{0.5cm}There is some research about the pacifying effect of international regime, UNCLOS, in the sea, while there are lack of research that analyzes maritime claims from the perspective of hegemonic power. This research tries to supplement this part. The results show that naval hegemon has an incentive to settle maritime claims peacefully. Therefore, under the powerful influences of naval hegemon, states are more likely use peaceful settlement strategies over maritime claims. Another interesting finding of this research is that the pacifying effect of naval hegemon varies based on alliance. It indicates that when one or two claimants are allied with naval hegemon, naval hegemon pays more attention to these maritime claims in order to avoid entrapment problem that leads to frequent peaceful settlement attempts. 

Although this research shows some meaningful results, it has some limitations. First, it does not capture the core concept of hegemonic stability theory. Based on the power transition theory and long cycle theory, decline of hegemonic power itself does not always lead to unstable situations. The wane of hegemonic power should be accompanied with the rise of challenging states based on different growth rates and technological innovations. However, this research only measure capability of naval hegemon, and this measurement does not reflect rising capabilities of challengers. Therefore, future research requires to reflect this dynamics or disparities of naval power between naval hegemon and challenging states. Along with this, separating levels such as global hegemon vs. global challenger or regional hegemon vs. regional challenger or global hegemon vs. regional hegemon would provide more insights about maritime claims. Probably, including these variables will provide some evidence to explain frequent occurrences of maritime claims in the South China Sea from the perspective of dynamics between the United States Navy (naval hegemon) and Chinese Navy (challenger). 

 Second, the geographic distance from the maritime claim and the position of naval hegemon also plays an important role in determining the pacifying effect of naval hegemon. Naturally, the farther from a state’s territory it attempts to project military capabilities, the harder it becomes to make efforts effective (Lemke, 2002). Therefore, future study needs to supplement a power degradation problem by adding a variable such as geographic distance between the position of maritime claims and naval hegemon’s naval installation

Third, this research includes any type of alliances with naval hegemon, but some types of alliance which focus on neutrality and nonaggression between states do not always lead to entrapment problem for naval hegemon. It other words, the extent of entrapment problem is different based on the types of alliance. Therefore, future research needs to separate the effect of alliance based on its type. 

Fourth, spread of violence from maritime claims to land territorial claims or from land
territorial claims to maritime claims would be interesting topic. In other words, how do maritime claims influence the occurrence or management of land/river claims? ICOW datasets include a variable that reflect the connection between territorial claims. So, including this variable would provide more explanations regarding territorial claims.

Lastly, in terms of regional scope, this study only covers the Western Hemisphere and Europe area. However, a lot of recent maritime issues occurs in Asian states, especially in the South China Sea. Therefore, including other regions such as Asia or Africa is conducive to expand the understanding regarding current maritime claims.

\end{document}

	 
